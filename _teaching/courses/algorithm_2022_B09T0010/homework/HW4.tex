% !Mode:: "TeX:UTF-8"
\documentclass[UTF8,16pt]{article} %文档声明  article,report,book

%导言区
\usepackage[space]{ctex} %引入宏包  
\usepackage{graphicx}
\usepackage{listings}
\usepackage{color}
\usepackage{lineno,hyperref,amsmath}
\usepackage[a4paper,left=35mm,right=35mm,top=45mm,bottom=15mm]{geometry} 

%opening
\title{《\heiti{算法设计与分析}》\heiti{第{\color{red}4}次作业}}
\author{\kaishu{姓名:}\underline{XXX} \quad\quad\quad\quad\quad  \kaishu{学号:}\underline{XXXXXXXX}}
\date{}

\begin{document}
	
\maketitle
\vbox{} %空行

\noindent{\heiti{题目1}}:从哈尔滨到上海的高速公路上有若干个加油站。如果汽车从一个加油站出发时油箱是满的,则汽车可以顺利到达下一个加油站。试设计一个汽车加油方案使得汽车在整个行驶路程中加油的次数最少,证明所给方案的正确性(需给出最优子结构和贪心选择性两方面的证明)。\\
{\heiti{答:}}\\
\rule[0pt]{14.3cm}{0.05em}

\vbox{} %空行
\noindent{\heiti{题目2}}:假设你是一位很棒的家长,想要给你的孩子们一些小饼干。但是,每个孩子最多只能给一块饼干。对每个孩子 $i$,都有一个胃口值 $g_{i}$,这是能让孩子们满足胃口的饼干的最小尺寸;并且每块饼干 $j$,都有一个尺寸 $s_{j}$。如果 $s_{j}$ $\ge$ $g_{i}$,我们可以将这个饼干 $j$ 分配给孩子 $i$ ,这个孩子会得到满足。你的目标是尽可能满足越多数量的孩子,并输出这个最大数值。注意:你可以假设胃口值为正。一个小朋友最多只能拥有一块饼干。试设计一个算法求解该问题,答案需包含以下内容:证明该问题的贪心选择性,描述算法思想并给出伪代码。示例如下:\\
输入: [1,2], [1,2,3]\\
输出: 2\\
解释: 你有两个孩子和三块小饼干,2个孩子的胃口值分别是1,2。你拥有的饼干数量和尺寸都足以让所有孩子满足。所以你应该输出2。\\
{\heiti{答:}}\\
\rule[0pt]{14.3cm}{0.05em}

\vbox{} %空行
\noindent{\heiti{题目3}}:现有一台计算机,在某个时刻同时到达了$n$个任务。该计算机在同一时间只能处理一个任务,每个任务都必须被不间断地得到处理。该计算机处理这$n$个任务需要的时间分别为$a_1, a_2,...a_n$。将第i个任务在调度策略中的结束时间记为$e_i$,请设计一个贪心算法输出这n个任务的一个调度使得用户的平均响应时间$\frac{\sum e_i}{n}$达到最小。试设计一个算法求解该问题,答案需包含以下内容:证明该问题的贪心选择性,描述算法思想并给出伪代码。\\
{\heiti{答:}}\\
\vbox{} %空行

\end{document}
