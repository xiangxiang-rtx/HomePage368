% !Mode:: "TeX:UTF-8"
\documentclass[UTF8,16pt]{article} %文档声明  article,report,book

%导言区
\usepackage[space]{ctex} %引入宏包
\usepackage{graphicx}
\usepackage{listings}
\usepackage{color}
\usepackage{lineno,hyperref,amsmath}
\usepackage[a4paper,left=35mm,right=35mm,top=45mm,bottom=15mm]{geometry}

%opening
\title{《\heiti{算法设计与分析}》\heiti{第{\color{red}3}次作业}}
\author{\kaishu{姓名:}\underline{XXX} \quad\quad\quad\quad\quad  \kaishu{学号:}\underline{XXXXXXXX}}
\date{}

\begin{document}
	
\maketitle
\vbox{} %空行


\noindent{\heiti{题目1}}:某公司生产长钢管,然后将钢条切割成不同的长度拿去售卖。不同长度的钢管售价不一样。钢管的长度售价表如下:
% Please add the following required packages to your document preamble:
% \usepackage{graphicx}
\begin{table}[htb]
	\resizebox{\textwidth}{!}{%
		\begin{tabular}{|c|c|c|c|c|c|c|c|c|c|c|}
			\hline
			长度i(米)        & 1 & 2 & 3 & 4 & 5  & 6  & 7  & 8  & 9  & 10 \\ \hline
			售价$P_{i}$ & 1 & 5 & 8 & 9 & 10 & 17 & 17 & 20 & 24 & 30 \\ \hline
		\end{tabular}%
	}
	\caption{钢管售价表}
	\label{tab:my-table}
\end{table}

\noindent
有一根长度为$n$的钢管,请设计一个动态规划算法进行切割,让公司的收益达到最大。注:若长度为n英寸的钢条的价格$P_{n}$足够大,最优解可能就是完全不需要切割。\\
\textbf{注意:}请给出算法思路、递推方程及其解释,并且分别运用自顶向下方法和自底向上的方法给出伪代码。\\
{\heiti{答:}}\\
\rule[0pt]{14.3cm}{0.05em}



\vbox{} %空行
\noindent{\heiti{题目2}}:买卖股票的最佳时机简单版:给定一个数组,它的第 i 个元素是一支给定股票第 i 天的价格。如果你最多只允许完成一笔交易(即买入和卖出一支股票一次),设计一个算法来计算你所能获取的最大利润。注意:你不能在买入股票前卖出股票。示例如下:\\
输入: [7,1,5,3,6,4] \\
输出: 5 \\
解释: 在第 2 天(股票价格 = 1)的时候买入,在第 5 天(股票价格 = 6)的时候卖出,最大利润 = 6-1 = 5 。注意利润不能是 7-1 = 6, 因为卖出价格需要大于买入价格。请设计一个时间复杂度为$O(n)$的算法。\\
\textbf{注意:}若使用动态规划,请给出算法思路、递推方程及其解释,并用伪代码描述算法;若不是使用动态规划,请给出算法思路、并用伪代码描述算法。\\
{\heiti{答:}}\\
\rule[0pt]{14.3cm}{0.05em}


\vbox{} %空行
\noindent{\heiti{题目3}}: $n$个作业${1,2,...n}$要在又2台机器$M_1$和$M_2$组成的流水线上完成加工。每个作业加工的顺序都是先在$M_1$上加工,然后在$M_2$上加工。$M_1$和$M_2$加工作业$i$所需的时间分别为$a_i$和$b_i$,$1\leq  i\leq n$。流水作业调度问题要求确定这$n$个作业的最优加工顺序,使得从第一个作业在机器$M_1$上开始加工,到最后一个作业在机器$M_2$上加工完成所需的时间最少。

直观上,一个最优调度应使机器$M_1$没有空闲时间,且机器$M_2$的空闲时间最少。在一般情况下,机器$M_2$上会有机器空闲和作业积压两种情况。

设全部作业的集合$N={1,2,..,n}, S\subseteq N$是$N$的作业子集。在一般情况下,机器$M_1$开始加工$S$中作业时,机器$M_2$还在加工其他作业,要等时间$t$后才可利用。将这种情况下完成$S$中作业所需的最短时间记为$T(S,t)$。流水作业调度问题的最优值为$T(N,0)$.\\
\textbf{注意:}
请认真阅读自学课本3.9节流水作业调度,完成以下任务:\\
(1)给出课本中两种不同动态规划算法思路(递归式+基于Johnson原则)的伪代码(注意是伪代码,不是课本上的实现代码)\\
(2)复现这两种思路,给出程序运行截图(要有输入,输出以及运行时间)\\
(3)枚举n,记录运行时间并画出n与运行时间的关系图\\


{\heiti{答:}}\\
\rule[0pt]{14.3cm}{0.05em}


\vbox{} %空行
\vbox{} %空行

%\begin{figure}[bh]
%	\includegraphics[width=1.0\textwidth]{time.jpg}
%	\caption{画出$n$与运行时间的关系图(示例)}
%\end{figure}




\end{document}
