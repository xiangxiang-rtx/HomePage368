% !Mode:: "TeX:UTF-8"
\documentclass[UTF8,16pt]{article} %文档声明  article,report,book

%导言区
\usepackage[space]{ctex} %引入宏包
\usepackage{graphicx}
\usepackage{listings}
\usepackage{color}
\usepackage{lineno,hyperref,amsmath}
\usepackage[a4paper,left=35mm,right=35mm,top=45mm,bottom=15mm]{geometry}

%opening
\title{《\heiti{算法设计与分析}》\heiti{第{\color{red}2}次作业}}
\author{\kaishu{姓名:}\underline{XXX} \quad\quad\quad\quad\quad  \kaishu{学号:}\underline{XXXXXXXX}}
\date{}

\begin{document}
	
\maketitle
\vbox{} %空行


\noindent{\heiti{题目1}}:求下列递推关系表示的算法复杂度。\\
(1)$T(n)=9T(\frac{n}{3}) + n$\\
(2)$T(n)=8T(\frac{n}{6})+n^{\frac{3}{2}}\log n$\\
(3)$T(n)=7T(\frac{n}{7})+n$\\
{\heiti{答:}}答案写在这里\\
\rule[0pt]{14.3cm}{0.05em}

\vbox{} %空行
\noindent{\heiti{题目2}}:设A[0:n-1]是一个元素个数为n的未排序的数组,运用分治算法找到第 k个最大的元素。请注意,你需要找的是数组排序后的第 k 个最大的元素,而不是第 k 个不同的元素。你需要给出具体的\textbf{算法思路}、\textbf{伪代码},并设计一个时间复杂度为$O(n)$的算法(可以是平均或最坏情况。若能清晰写出最坏情况复杂度为$O(n)$的算法,有额外分)。\\
{\heiti{答:}}答案写在这里\\
\rule[0pt]{14.3cm}{0.05em}

\vbox{} %空行
\noindent{\heiti{题目3}}:动手设计并实现一个可以进行两个$n$位大整数的乘法运算的算法。你需要给出具体的\textbf{算法思路}、\textbf{伪代码},并对你设计的算法进行\textbf{复杂度分析},除此之外,你还需要给出\textbf{算法运行结果截图},并用你熟悉的图形库\textbf{画出输入规模$n$与运行时间的关系图}。\\
{\heiti{答:}}答案写在这里\\
\rule[0pt]{14.3cm}{0.05em}

%\begin{figure}[bh]
%\includegraphics[width=1.0\textwidth]{time.jpg} %把图片路径放这里
%\caption{输入规模$n$与运行时间的关系图(示例)}
%\end{figure}


\vbox{} %空行
\vbox{} %空行


\end{document}
