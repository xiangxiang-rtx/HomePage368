% !Mode:: "TeX:UTF-8"

\documentclass[UTF8,16pt]{article} %文档声明  article,report,book

%导言区
\usepackage[space]{ctex} %引入宏包
\usepackage{graphicx}
\usepackage{listings}
\usepackage{color}
\usepackage{lineno,hyperref,amsmath}
\usepackage[a4paper,left=35mm,right=35mm,top=45mm,bottom=15mm]{geometry}

%opening
\title{《\heiti{算法设计与分析}》\heiti{第{\color{red}3}次作业}}
\author{\kaishu{姓名:}\underline{XXX} \quad\quad\quad\quad\quad  \kaishu{学号:}\underline{XXXXXXXX}}
\date{}

\begin{document}
	
\maketitle
\vbox{} %空行

\section*{\textbf{算法分析题}}
\noindent{\heiti{题目1}}:给出$N$个$1-9$ 的数字$(v_{1},v_{2},...,v_{N})$,不改变它们的相对位置,在中间加入$K$ 个乘号和$N-K-1$ 个加号,(括号随便加)使最终结果尽量大。因为乘号和加号一共就是$N-1$ 个了,所以恰好每两个相邻数字之间都有一个符号。请给出算法思路、递推方程及其解释,并用伪代码描述算法。\\
例如:$N=5, K=2$,5个数字分别为1、2、3、4、5,可以加成:\\
$1*2*(3+4+5)=24$ \\
$1*(2+3)*(4+5)=45$ \\
$(1*2+3)*(4+5)=45$ \\
{\heiti{答:}}\\
\rule[0pt]{14.3cm}{0.05em}

\vbox{} %空行
\noindent{\heiti{题目2}}:在自然语言处理中一个重要的问题是分词,例如句子“他说的确实在理”中“的确”“确实”“实在”“在理”都是常见的词汇,但是计算机必须为给定的句子准确判断出正确分词方法。一个简化的分词问题如下:给定一个长字符串$y=y_{1}y_{2}...y_{n}$,分词是把$y$切分成若干连续部分,每部分都单独成为词汇。我们用函数$quality(x)$ 判断切分后的某词汇$x=x_{1}x_{2}...x_{k}$ 的质量,函数值越高表示该词汇的正确性越高。分词的好坏用所有词汇的质量的和来表示。例如对句子“确实在理”分词,$quality$(确实) + $quality$(在理) $>$ $quality$(确)+$quality$(实在)+$quality$(理)。请设计一个动态规划算法对字符串$y$ 分词,要求最大化所有词汇的质量和。(假定你可以调用$quality(x)$ 函数在一步内得到任何长度的词汇的质量),请给出算法思路、递推方程及其解释,并用伪代码描述算法。 \\
{\heiti{答:}}\\
\rule[0pt]{14.3cm}{0.05em}

\vbox{} %空行
\noindent{\heiti{题目3}}:买卖股票的最佳时机简单版:给定一个数组,它的第 i 个元素是一支给定股票第 i 天的价格。如果你最多只允许完成一笔交易(即买入和卖出一支股票一次),设计一个算法来计算你所能获取的最大利润。注意:你不能在买入股票前卖出股票。示例如下:\\
输入: [7,1,5,3,6,4] \\
输出: 5 \\
解释: 在第 2 天(股票价格 = 1)的时候买入,在第 5 天(股票价格 = 6)的时候卖出,最大利润 = 6-1 = 5 。注意利润不能是 7-1 = 6, 因为卖出价格需要大于买入价格。\\
(1) 请设计一个时间复杂度为$O(n^2)$ 的算法。\\
(2) 请设计一个时间复杂度为$O(n)$ 的算法。\\
注意:若使用动态规划,请给出算法思路、递推方程及其解释,并用伪代码描述算法;若不是使用动态规划,请给出算法思路、并用伪代码描述算法。\\
\noindent
{\heiti{答:}}\\
\rule[0pt]{14.3cm}{0.05em}

\vbox{} %空行
\vbox{} %空行
\section*{\textbf{算法实现题}}
\noindent{\heiti{题目1}}:给定一个拥有正整数和负整数的二维数组,子矩形是位于整个数组中的任何大小为1*1 或更大的连续子数组。矩形的总和是该矩形中所有元素的总和。在这个问题中,具有最大和的子矩形称为最大子矩形。请求出二维数组中的最大子矩阵之和。\\
\indent 题目细节及提交地址:\url{https://vjudge.net/contest/363101};源码使用在线提交方式,提交密码:seu711184;用户名使用学号-姓名格式。\\
\noindent
{\heiti{答:}}\\

\noindent
{\heiti{最优子结构:}}\\

\noindent
{\heiti{递推公式:}}\\


\noindent
{\heiti{用一个5*5的二维数组实例说明解题过程:}}\\

\noindent
{\heiti{结果截图:}}\\

\end{document}
