% !Mode:: "TeX:UTF-8"
\documentclass[UTF8,16pt]{article} %文档声明  article,report,book

%导言区
\usepackage[space]{ctex} %引入宏包  
\usepackage{graphicx}
\usepackage{listings}
\usepackage{color}
\usepackage{lineno,hyperref,amsmath}
\usepackage[a4paper,left=35mm,right=35mm,top=45mm,bottom=15mm]{geometry} 

%opening
\title{《\heiti{算法设计与分析}》\heiti{第{\color{red}4}次作业}}
\author{\kaishu{姓名:}\underline{XXX} \quad\quad\quad\quad\quad  \kaishu{学号:}\underline{XXXXXXXX}}
\date{}

\begin{document}
	
\maketitle
\vbox{} %空行

\section*{\textbf{算法分析题}}
\noindent{\heiti{题目1}}:给定n个物品,物品价值分别为$P_{1},P_{2},...,P_{n}$,物品重量分别为$W_{1},W_{2},...,W_{n}$,背包容量为$M$。每种物品可部分装入到背包中。输出$X_{1},X_{2},...,X_{n}$, $0\le X_{i}\le 1$, 使得$\sum_{1\le i \le n}P_{i}X_{i}$最大,且$\sum_{1\le i\le n}W_{i}X_{i}\le M$。试设计一个算法求解该问题,答案需包含以下内容:证明该问题的贪心选择性,描述算法思想并给出伪代码。\\
{\heiti{答:}}\\
\rule[0pt]{14.3cm}{0.05em}

\vbox{} %空行
\noindent{\heiti{题目2}}:假设你是一位很棒的家长,想要给你的孩子们一些小饼干。但是,每个孩子最多只能给一块饼干。对每个孩子 $i$,都有一个胃口值 $g_{i}$,这是能让孩子们满足胃口的饼干的最小尺寸;并且每块饼干 $j$,都有一个尺寸 $s_{j}$。如果 $s_{j}$ $\ge$ $g_{i}$,我们可以将这个饼干 $j$ 分配给孩子 $i$ ,这个孩子会得到满足。你的目标是尽可能满足越多数量的孩子,并输出这个最大数值。注意:你可以假设胃口值为正。一个小朋友最多只能拥有一块饼干。试设计一个算法求解该问题,答案需包含以下内容:证明该问题的贪心选择性,描述算法思想并给出伪代码。示例如下:\\
输入: [1,2], [1,2,3]\\
输出: 2\\
解释: 你有两个孩子和三块小饼干,2个孩子的胃口值分别是1,2。你拥有的饼干数量和尺寸都足以让所有孩子满足。所以你应该输出2。\\
{\heiti{答:}}\\
\rule[0pt]{14.3cm}{0.05em}

\vbox{} %空行
\noindent{\heiti{题目3}}:给定一个区间的集合,找到需要移除区间的最小数量,使剩余区间互不重叠。可以认为区间的终点总是大于它的起点, 另外像区间 [1,2] 和 [2,3] 这样边界相互“接触”的区间,可以认为没有相互重叠。试设计一个算法求解该问题,答案需包含以下内容:证明该问题的贪心选择性,描述算法思想并给出伪代码。示例如下:\\
输入: [ [1,2], [2,3], [3,4], [1,3] ]\\
输出: 1\\
解释: 移除 [1,3] 后,剩下的区间没有重叠。\\
{\heiti{答:}}\\
\vbox{} %空行

\end{document}
